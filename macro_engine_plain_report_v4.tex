\documentclass[11pt]{article}
\usepackage[margin=1in]{geometry}
\usepackage[hidelinks]{hyperref}
\usepackage{booktabs}
\usepackage{longtable}
\usepackage{array}
\usepackage{enumitem}
\setlength{\parindent}{0pt}
\setlength{\parskip}{0.45em}

\title{Macro Risk V4 Release Review and Technical Record}
\author{}
\date{2026-02-20}

\begin{document}
\maketitle

\section{Outline and purpose of this report}
This report is structured to answer six practical questions:
\begin{enumerate}[leftmargin=1.4em]
  \item What is this macro engine as a complete system: assumptions, methodology, and design.
  \item What are the V4 validation results, and what do they imply for release and promotion.
  \item What changed in V4, and why those changes matter.
  \item Is V4 materially better than V3 on governance outcomes.
  \item What risks remain, and how serious they are.
  \item What operating discipline is required after release.
\end{enumerate}

All metrics in this report are sourced from package artifacts under
\texttt{outputs/macro\_engine/validation/}.

\section{What this macro engine does as a whole}
\subsection{System objective}
The engine produces quarterly macro paths, interval forecasts, and governance outputs
used by downstream risk processes.

\subsection{Data and assumption framework}
Core data artifacts:
\begin{itemize}[leftmargin=1.2em]
  \item \texttt{data/macro\_panel\_quarterly\_raw.csv}
  \item \texttt{data/macro\_panel\_quarterly\_model.csv}
  \item \texttt{data/macro\_panel\_metadata.json}
\end{itemize}

Assumption design combines:
\begin{itemize}[leftmargin=1.2em]
  \item short-horizon empirical forecasting,
  \item bridge dynamics,
  \item long-horizon scenario overlays.
\end{itemize}

\subsection{Methodology and architecture summary}
\begin{itemize}[leftmargin=1.2em]
  \item Forecast horizon: 80 quarters.
  \item Incumbent regime (\texttt{champion\_a}): short Q1--Q12, bridge Q13--Q24, long Q25--Q80.
  \item Challenger regime (\texttt{champion\_b}): short Q1--Q16, bridge Q17--Q28, long Q29--Q80.
  \item Candidate short-horizon models: BVAR, AR, RW.
  \item Champion selection buckets: Q1..Q4 and Q5..Q12.
\end{itemize}

\subsection{Release handoff outputs}
Handoff set:
\begin{itemize}[leftmargin=1.2em]
  \item \texttt{outputs/macro\_engine/pd\_regressors\_forecast\_levels.csv}
  \item \texttt{outputs/macro\_engine/pd\_regressors\_forecast\_derived.csv}
  \item \texttt{outputs/macro\_engine/pd\_regressors\_metadata.json}
\end{itemize}

Primary targets:
\texttt{unemployment\_rate}, \texttt{ust10\_rate}, \texttt{hpi\_yoy}
(mapped from \texttt{hpi\_growth\_yoy}).

\section{V4 validation results and release implications}
Source:
\texttt{outputs/macro\_engine/validation/validation\_summary.json}

\subsection{Release profile results}
\begin{longtable}{@{}p{0.36\textwidth}p{0.14\textwidth}p{0.14\textwidth}p{0.14\textwidth}p{0.12\textwidth}@{}}
\toprule
Metric & Threshold & Incumbent & Challenger & Result \\
\midrule
Release pass & required & True & True & PASS \\
Minimum required-cell \(n\_{oos}\) & \(\ge 40\) & 44 & 40 & PASS \\
Median rRMSE h1..h2 & \(\le 1.00\) & 0.9744 & 1.0000 & PASS \\
Median rRMSE h3..h4 & \(< 0.98\) & 0.9208 & 0.9333 & PASS \\
Mean CRPS gain h5..h12 vs RW & \(\ge 3.0\%\) & 15.1304\% & 12.7381\% & PASS \\
Coverage90 pass-rate & \(\ge 0.75\) & 0.9444 & 1.0000 & PASS \\
Width ratio mean & \(\le 1.35\) & 1.3438 & 1.2748 & PASS \\
Width ratio per-variable max & \(\le 1.60\) & 1.4779 & 1.3767 & PASS \\
Boundary consistency & pass & True & True & PASS \\
Scenario checks & pass & True & True & PASS \\
\bottomrule
\end{longtable}

\subsection{Promotion results}
\begin{longtable}{@{}p{0.52\textwidth}p{0.22\textwidth}p{0.16\textwidth}@{}}
\toprule
Metric & Threshold & Actual \\
\midrule
Promotion pass & pass & True \\
CRPS gain h9..h12 (challenger vs incumbent) & \(\ge 5.0\%\) & 14.1399\% \\
Short-horizon CRPS worsen h1..h4 & \(\le 1.0\%\) & -8.6351\% \\
Boundary comparator pass & required & True \\
\bottomrule
\end{longtable}

\subsection{Implication for release and promotion}
The result is not borderline.
Release and promotion both pass with meaningful margin on medium-horizon gain,
while short-horizon behavior improves rather than worsens.

\section{What changed in V4 and why it mattered}
The following changes were material to governance outcomes:
\begin{enumerate}[leftmargin=1.4em]
  \item Challenger evaluation path was corrected to use challenger champion artifacts end-to-end.
  \item Scenario timing checks were aligned with regime start logic.
  \item Small-gap tie-break logic was upgraded to use coverage-width balance.
  \item Challenger ust10 long-bucket calibration constraints were tightened.
  \item Calibration monotonicity checks were added.
  \item Boundary consistency checks were added to gate evaluation.
\end{enumerate}

Why this mattered:
these changes improved challenger quality and interval discipline,
which directly addressed the conditions that previously blocked promotion.

\section{Is V4 materially better than V3 on governance}
\subsection{Direct comparison (V3 vs V4)}
\begin{longtable}{@{}p{0.42\textwidth}p{0.18\textwidth}p{0.18\textwidth}p{0.12\textwidth}@{}}
\toprule
Metric & V3 & V4 & Direction \\
\midrule
Challenger release pass & False & True & Improved \\
Promotion pass & False & True & Improved \\
CRPS gain h9..h12 (challenger vs incumbent) & 1.6426\% & 14.1399\% & Improved \\
Short-horizon CRPS worsen h1..h4 & 0.4207\% & -8.6351\% & Improved \\
Challenger width ratio mean & 1.5285 & 1.2748 & Improved \\
Coverage-fail count & 5 & 2 & Improved \\
\bottomrule
\end{longtable}

\subsection{Conclusion from comparison}
On governance metrics that drive release and promotion decisions,
V4 is materially stronger than V3.

\section{Remaining risks and severity}
\subsection{Known residual issue}
Current coverage-fail cells:
\begin{itemize}[leftmargin=1.2em]
  \item \texttt{hpi\_growth\_yoy}, h=6, coverage90=0.7727, width-ratio=1.4147
  \item \texttt{hpi\_growth\_yoy}, h=11, coverage90=0.7955, width-ratio=1.7148
\end{itemize}

\subsection{Severity assessment}
\begin{itemize}[leftmargin=1.2em]
  \item Scope: localized to one variable family and medium bucket.
  \item Gate impact: no release or promotion failure in current state.
  \item Severity classification: \textbf{medium, controlled}.
\end{itemize}

\section{Next steps}
\begin{enumerate}[leftmargin=1.4em]
  \item Keep V3 available as rollback benchmark.
  \item Move long-run structural redesign work to V5 scope.
\end{enumerate}

\section{Conclusion and decision}
\textbf{V4 for controlled release.}

Reasons:
\begin{itemize}[leftmargin=1.2em]
  \item release gates pass for incumbent and challenger,
  \item promotion gate passes,
  \item no-regression checks pass,
  \item residual risk is narrow, explicit, and monitorable.
\end{itemize}

\section{Reproducibility commands}
\begin{verbatim}
python3 scripts/fetch_macro_panel_fred.py \
  --raw-output data/macro_panel_quarterly_raw.csv \
  --model-output data/macro_panel_quarterly_model.csv \
  --metadata-output data/macro_panel_metadata.json

python3 scripts/run_macro_forecast_engine.py \
  --config macro_engine_config.json \
  --output-dir outputs/macro_engine

python3 scripts/run_macro_validation.py \
  --config macro_engine_config.json \
  --input data/macro_panel_quarterly_model.csv \
  --output-dir outputs/macro_engine/validation \
  --champion-map-output outputs/macro_engine/champion_map.json \
  --verbose-validation

python3 scripts/export_pd_macro_subset.py \
  --config macro_engine_config.json \
  --input outputs/macro_engine/macro_forecast_paths.csv \
  --model-panel data/macro_panel_quarterly_model.csv \
  --levels-output outputs/macro_engine/pd_regressors_forecast_levels.csv \
  --derived-output outputs/macro_engine/pd_regressors_forecast_derived.csv \
  --metadata-output outputs/macro_engine/pd_regressors_metadata.json
\end{verbatim}

\end{document}
